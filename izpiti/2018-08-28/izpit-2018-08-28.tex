\documentclass[arhiv]{../izpit}
\usepackage{fouriernc}
\usepackage{xcolor}
\usepackage{tikz}
\usepackage{fancyvrb}
\VerbatimFootnotes{}

\begin{document}

\izpit{Programiranje I: 3.\ izpit}{28.\ avgust 2018}{
  Čas reševanja je 150 minut.
  Veliko uspeha!
}

%%%%%%%%%%%%%%%%%%%%%%%%%%%%%%%%%%%%%%%%%%%%%%%%%%%%%%%%%%%%%%%%%%%%%%%
\naloga[]

\podnaloga
Napišite funkcijo
\begin{verbatim}
    razlika_kvadratov : int -> int -> int,
\end{verbatim}
ki za dani števili izračuna razliko med kvadratom vsote in vsoto kvadratov.

\podnaloga
Napišite funkcijo
\begin{verbatim}
    uporabi_na_paru : ('a -> 'b) -> 'a * 'a -> 'b * 'b,
\end{verbatim}
ki podano funkcijo uporabi na vsakem elementu para.

\podnaloga
Napišite funkcijo
\begin{verbatim}
    ponovi_seznam : int -> 'a list -> 'a list,
\end{verbatim}
za katero \verb|ponovi_seznam n sez| vrne seznam sestavljen iz \verb|n| ponovitev podanega seznama.
Če je \verb|n| manjši ali enak $0$, naj funkcija vrne prazen seznam.

\podnaloga
Napišite funkcijo
\begin{verbatim}
    razdeli : int list -> int list * int list,
\end{verbatim}
za katero \verb|razdeli sez| vrne par seznamov, kjer levi seznam v paru vsebuje vse negativne elemente \verb|sez|, desni pa preostale.
Za vse točke naj bo funkcija \emph{repno-rekurzivna}.

%%%%%%%%%%%%%%%%%%%%%%%%%%%%%%%%%%%%%%%%%%%%%%%%%%%%%%%%%%%%%%%%%%%%%%%
\naloga
Napišite funkcijo
\begin{verbatim}
    monotona_pot : int drevo -> int list,
\end{verbatim}
ki v danem dvojiškem drevesu poišče najdaljšo pot, v kateri vrednosti v vozliščih bodisi naraščajo bodisi padajo. Na primer, v spodnjem levem drevesu je to pot $3 - 10 - 13 - 14$, v desnem pa $20 - 10 - 7 - 4 - 1$:

\[
  \begin{tikzpicture}[level distance=0.9cm,
    level 1/.style={sibling distance=4cm},
    level 2/.style={sibling distance=2cm},
    level 3/.style={sibling distance=2cm}
    ]
    \node {11}
      child {node {\textbf{10}}
        child {node {\textbf{3}}}
        child {node {\textbf{13}}
          child {node {\textbf{14}}}
          child {node {6}}
        }
      }
      child {node {8}
        child {node {2}}
        child {node {10}}
      };
  \end{tikzpicture}
  \qquad\qquad
  \begin{tikzpicture}[level distance=0.9cm,
    level 1/.style={sibling distance=4cm},
    level 2/.style={sibling distance=2cm},
    level 3/.style={sibling distance=2cm}
    ]
    \node {\textbf{7}}
      child {node {\textbf{10}}
        child {node {\textbf{20}}}
        child {node {9}
          child {node {15}}
          child {node {6}}
        }
      }
      child {node {\textbf{4}}
        child {node {\textbf{1}}}
        child {node {10}}
      };
  \end{tikzpicture}
\]
\prostor

%%%%%%%%%%%%%%%%%%%%%%%%%%%%%%%%%%%%%%%%%%%%%%%%%%%%%%%%%%%%%%%%%%%%%%%
\naloga[]
\newcommand{\ostalo}[1]{\big[ #1 \big]}
\newcommand{\filter}[2]{\ostalo{ #2 }_{#1}}
\newcommand{\nic}{\;{\cdot}\;}
Konstruiramo podatkovni tip verige filtrov:
\begin{verbatim}
  type 'a veriga = | Filter of ('a -> bool) * 'a list * 'a veriga
                   | Ostalo of 'a list,
\end{verbatim}
kjer \verb|Filter(f, sez, rep)| predstavlja člen v verigi, ki v seznamu \verb|sez| hrani elemente,
za katere funkcija \verb|f| vrne \verb|true|, parameter \verb|rep| pa hrani preostanek verige.

\podnaloga
Napišite prazen filter (torej takšen, ki še ne hrani nobenega elementa) \verb|test : int veriga|,
ki v prvem členu hrani negativna števila, v drugem členu hrani števila manjša od 10 in
v tretjem členu preostala števila.
\[
  \filter{< 0}{\nic} \sim \filter{< 10}{\nic} \sim \ostalo{\nic}
\]

\podnaloga
Napišite funkcijo \verb|vstavi : 'a -> 'a veriga -> 'a veriga|,
ki vrne verigo, v kateri je dani element vstavljen na začetek seznama tistega člena verige, ki prvi zadošča filtracijski funkciji.
Pri vstavljanju števil $-5$, $7$, $100$, $-7$ in $2$ v zgornjo verigo bi dobili:
\[
  \filter{< 0}{{-5}, -7} \sim \filter{< 10}{7, 2} \sim \ostalo{100}
\]

\podnaloga
Napišite funkcijo \verb|poisci : 'a -> 'a veriga -> bool|,
ki preveri, ali je element vsebovan v katerem od seznamov v verigi filtrov. Pri tem naj preveri zgolj en seznam.

\podnaloga
Napišite funkcijo \verb|izprazni_filtre : 'a veriga -> 'a veriga * 'a list|,
ki kot par vrne verigo filtrov, kjer imajo vsi členi prazne sezname, in seznam vseh
elementov, ki so bili hranjeni v filtrih. Na primer, za zgornjo verigo bi poleg prazne verige funkcija vrnila še seznam $[{-5}, -7, 7, 2, 100]$.

\podnaloga
Napišite funkcijo \verb|dodaj_filter : ('a -> bool) -> 'a veriga -> 'a veriga|,
ki na začetek podane verige filtrov doda nov filter s filtracijsko funkcijo \verb|f|.
Elementi v vrnjeni verigi filtrov naj bodo pravilno razporejeni tudi glede na novo dodani člen. Na primer, če bi zgornji verigi dodali še filtracijsko funkcijo, ki preverja sodost, bi dobili:
\[
  \filter{\text{je sod}}{2, 100} \sim \filter{< 0}{{-5}, -7} \sim \filter{< 10}{7} \sim \ostalo{\nic}
\]


\end{document}

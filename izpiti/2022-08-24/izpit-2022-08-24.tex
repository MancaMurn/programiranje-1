\documentclass[arhiv]{../izpit}
\usepackage{fouriernc}
\usepackage{xcolor}
\usepackage{fancyvrb}


\begin{document}
	
\izpit{Programiranje I: 3. izpit}{24.\ avgust 2022}{
  Čas reševanja je 120 minut.
  Veliko uspeha!
}

%%%%%%%%%%%%%%%%%%%%%%%%%%%%%%%%%%%%%%%%%%%%%%%%%%%%%%%%%%%%%%%%%%%%%%%

\naloga

\podnaloga
Napišite funkcijo \verb|zamenjaj : ('a * 'b) * ('c * 'd) -> ('a * 'c) * ('b * 'd)|, ki sprejme dva para in vrne par parov, kjer je druga komponenta prvega para zamenjana s prvo komponento drugega.
  
\podnaloga
 Napišite funkcijo \verb|modus : int * int * int -> int option|, ki vrne modus 
  (najpogostejšo vrednost) podane trojice, ali pa \verb|None|, če so vse vrednosti 
  različne.
  
\podnaloga
 Napišite funkcijo \verb|uncons : 'a list -> ('a * 'a list) option|, ki seznam razdeli
  na potencialno glavo in rep.

\podnaloga
 Napišite funkcijo \verb|vstavljaj : 'a -> 'a list -> 'a list|, ki vstavi prvi 
  argument med vse ostale elemente seznama.

\begin{verbatim}
# vstavljaj 0 [1;2;3;4]
- : int list
[1;0;2;0;3;0;4]
\end{verbatim}

\podnaloga
Napišite funkcijo \verb|popolnoma_obrni : 'a list list -> 'a list list|, ki popolnoma obrne gnezden seznam (obrne vrstni red seznamov in elemente v podseznamih).
Funkcija mora biti repno rekurzivna in ne sme uporabljati funkcij iz standardne knjižnice.

\begin{verbatim}
# popolnoma_obrni [[1;2;3]; [4;5]; [7;8;9]]
- : int list list = [[9; 8; 7]; [5; 4]; [3; 2; 1]]
\end{verbatim}

%%%%%%%%%%%%%%%%%%%%%%%%%%%%%%%%%%%%%%%%%%%%%%%%%%%%%%%%%%%%%%%%%%%%%%%

\naloga

Neprazen označen trak stroja, ki upravlja z vrednostmi tipa \verb|'a|, predstavimo s tipom \verb|'a tape|, ukaze glavi, ki je nad označenim delom traku, pa s tipom \verb|'a command|.

\begin{verbatim}
type 'a tape = Tape of { left : 'a list; head : 'a; right : 'a list }

type 'a command = Left | Do of ('a -> 'a) | Right

let example = Tape { left = [ 3; 2; 1 ]; head = 4; right = [ 5; 6 ] }
\end{verbatim}

Spremenljivka \verb|example| predstavlja spodnjo konfiguracijo

\begin{verbatim}
         V
  1 2 3  4  5 6
\end{verbatim}

kjer je glava nad poljem z vrednostjo 4 (in indeksom 3), levo od glave so elementi \verb|3, 2, 1| in desno elementa \verb|5, 6|.


\podnaloga
Napišite funkcijo \verb|map : 'a tape -> ('a -> 'b) -> 'b tape|, ki sprejme trak in funkcijo, ter preslika elemente v traku. Ob koncu mora biti glava na enakem mestu (čeprav se vrednost pod glavo lahko spremeni)

\podnaloga
Napišite funkcijo \verb|izvedi : 'a tape -> 'a command -> 'a tape option|, ki sprejme trak in ukaz ter vrne nov trak, kjer je glava ustrezno premaknjena glede na ukaz.
Če je ukaz premik \verb|Left| ali \verb|Right|, se glava ustrezno premakne levo ali desno.
Če je ukaz \verb|Do|, potem aplicira funkcijo na element pod glavo, glava pa ostane na enakem mestu. Če je premik neveljaven (npr.~premik prek roba traku), potem funkcija vrne \verb|None|.

\begin{verbatim}
# izvedi example Left
- : tape option
Some (Tape { left = [ 2; 1 ]; head = 3; right = [ 4; 5; 6 ] })
\end{verbatim}

\podnaloga
Napišite funkcijo \verb|izvedi_ukaze : 'a tape -> 'a command list -> 'a tape|, ki sprejme trak in seznam ukazov in vrne nov posodobljen trak.
Posodabljanje traku premika glavo glede na ukaze (levo ali desno) ali pa aplicira funkcijo podano v \verb|Do| ukazu.
Če je kak ukaz neveljaven naj funkcija preneha z izvajanjem in vrne kar je bilo spremenjeno do sedaj.

\begin{verbatim}
# izvedi_ukaze example [Do ((+) 1); Left; Do ((+) 4); Right; Do ((+) 1); Right]
- : int tape =
Tape {left = [6; 7; 2; 1]; head = 5; right = [6]}
\end{verbatim}

\podnaloga
Napišite funkcijo \\\verb|naberi_in_pretvori : 'a tape -> 'a command list -> ('a * 'a) list * 'a tape|, ki sprejme trak in seznam ukazov. 
Funkcija ukaze izvaja enega za drugim, kjer najprej izvede ukaz, in nato za vsak ukaz \verb|Do| v seznam parov shrani vrednost pod glavo \textbf{pred in po izvedbi ukaza}.
Poleg seznama nabranih parov naj funkcija vrne tudi končno stanje traku.
Če med izvajanjem pride do napake, naj funkcije vrne do sedaj nabrane vrednosti in zadnje veljavno stanje traku. 

\begin{verbatim}
# naberi example [Do ((+) 1); Left; Do (( * ) 2); Right; Right; Right; Do ((-) 1); 
Right; Do (fun x -> x / 2 )]
- : (int * int) list * int tape =
([(4, 5); (3, 6); (6, -5)],
Tape {left = [5; 5; 6; 2; 1]; head = -5; right = []})
\end{verbatim}

\podnaloga
Napišite funkcijo \verb|pripravi_ukaze: 'a tape -> ('a -> 'a) -> ('a -> 'a) command list|, ki sprejme trak in funkcijo in vrne tak seznam ukazov, da za pravilno funkcijo \verb|map| velja \\
\verb|pripravi_ukaze tape f = lst => map tape f == izvedi_ukaze tape lst|.


%%%%%%%%%%%%%%%%%%%%%%%%%%%%%%%%%%%%%%%%%%%%%%%%%%%%%%%%%%%%%%%%%%%%%%%

\naloga

\emph{Nalogo lahko rešujete v Pythonu ali OCamlu.}

Miha bo izdal svojo kolekcijo NeZamenljivih Žetonov (NZŽ-jev), ki bodo upodabljali abstraktne slike.
Vsak žeton predstavlja lastništvo slike, ki za osnovo vzame bel pas dolžine $N$ pikslov. 
Miha lahko na ta pas drugega za drugim riše vzorce celoštevilske dolžine vsaj $1$ in največ $k$ pikslov, kjer je med dvema zaporednima vzorcema razdalja vsaj $l$ pikslov.

  Ker mu programiranje ne gre, ga zanima, koliko različnih NZŽ-jev lahko naredi za podano dolžino pasu $N$ ter parametra $k$ in $l$.
  Napišite funkcijo \verb|stevilo_nzz : int -> int -> int -> int| ali \verb|def nzz(n: int, k: int, l: int) -> int|, ki sprejme dolžino pasu $N$ in parametra $k$ in $l$ ter vrne število različnih slik, ki jih Miha lahko naredi.

Spodaj je narisanih\footnote{Slik ne prerisujte, saj so njihove vrednosti ocenjene na med 250.000 in 470.000 evrov.} vseh $11$ možnih slik dolžine $4$ in parametroma $k = 3$ in $l = 2$.
\begin{verbatim}
    _ _ _ _
    
    X _ _ _
    _ X _ _
    _ _ X _
    _ _ _ X
    
    X X _ _
    _ X X _
    _ _ X X
    
    X X X _
    _ X X X
    
    X _ _ X
\end{verbatim}
\end{document}
